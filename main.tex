%\documentclass{theme/franska}
\documentclass[a4paper,margin=3.25cm]{article}
%\documentclass[a4paper,margin=3.25cm]{article}
%[a4paper,margin=3.25cm]{geometry}


% Filler text
\usepackage{lipsum}
\usepackage{verbatim}
% Footer and header
\usepackage{fancyhdr}

%\setcounter{secnumdepth}{4} % how many sectioning levels to assign numbers to
%\setcounter{tocdepth}{4}    % how many sectioning levels to show in ToC


% Allow bilagor
\usepackage{standalone}

%nice tables
\usepackage{booktabs}
\renewcommand{\arraystretch}{1.2}


%bibliography packages
\usepackage[sorting=ynt]{biblatex}
\addbibresource{sample.bib}
\usepackage[nottoc]{tocbibind}

% Language package
\usepackage[utf8]{inputenc}
\usepackage[english,swedish]{babel}

%Quotations
\usepackage{csquotes}


%Images
\usepackage{graphicx}
\graphicspath{ {bilder/} }

%%Count Words





\usepackage{fancyhdr}

% Header and footer
	\pagestyle{fancy}
	% Clear headers and footers
	\fancyhead{}
	\fancyfoot{}
	% Remove line in the header
	\renewcommand{\headrulewidth}{0pt}
	% Draft, header
	\fancyhead[C]{--- Utkast ---}
	% Pagenumber, footer
	\fancyfoot[C]{\thepage}
	% Git commit data, footer
	\fancyfoot[R]{{\tiny Rev: \git}}

%\fancyhead{}
%\fancyhead[RO,LE]{Thesis Title}
%\fancyfoot{}
%\fancyfoot[LE,RO]{\thepage}
%\fancyfoot[LO,CE]{Chapter \thechapter}
%\fancyfoot[CO,RE]{Author Name}






\begin{document}

	% Set main language
	\selectlanguage{swedish}

	% Frontpage

	%Första sidan räknas inte
    \clearpage\thispagestyle{empty}\addtocounter{page}{-1}

	\begin{titlingpage}
\begin{SingleSpace}
\calccentering{\unitlength}
%\begin{adjustwidth*}{\unitlength}{-\unitlength}
\vspace*{13pt}
\begin{center}
\rule[0.5ex]{\linewidth}{2pt}\vspace*{-\baselineskip}\vspace*{3.2pt}
\rule[0.5ex]{\linewidth}{1pt}\\[\baselineskip]
{\HUGE Franska Skolan }\\[4mm]
{\Large \textit{Gymnasie Arbete}}\\
\rule[0.5ex]{\linewidth}{1pt}\vspace*{-\baselineskip}\vspace{3.2pt}
\rule[0.5ex]{\linewidth}{2pt}\\
\vspace{6.5mm}
{\large Av}\\
\vspace{6.5mm}
{\large\textsc{Getz Mikalsen}}\\
\vspace{11mm}
\includegraphics[scale=0.6]{logo}\\
\vspace{6mm}
{\large Gymnasie Arbete\\
\textsc{Franska skolan}}\\
\vspace{11mm}
\begin{minipage}{10cm}
%%DUMMY TEXT HERE
\lipsum[66]
%%DUMMY TEXT ENDS HERE
\end{minipage}\\
\vspace{9mm}
\today        %%Date of compilation
%%{\large\textsc{Oktober 2014}}  Other date
\vspace{12mm}
\end{center}
%\begin{flushright}
%{\small Antal ord: \wordcount}
%\end{flushright}
%\end{adjustwidth*}
\end{SingleSpace}

\end{titlingpage}

\pagebreak




	% Select English as language before the Abstract
	\selectlanguage{english}
	\begin{abstract}
%Our report today has been divided into two separate portions. The discussion of nutmeg and its composition, and of the possible involvement of its chemical components. The psychotropic intoxication has a natural division into two areas of presentation. The first is a brief description of the plant; a presentation of the methods and procedures for the isolation and the identification of the many components in the oil from the plant, and a careful definition of those components that are most probably involved in the intoxicative syndrome.

The extension of these components in to the corresponding amphetamines, their effectiveness in humans, and the likelihood of their being an acceptable explanation of the effects of the total nutmeg, will constitute the latter part of this report. Also included is some of the history of nutmeg, and a description of the style and extent of its usage in various cultures. In the reports that will follow, specific descriptions of the human syndrome of intoxication, and some of the pharmacological ramifications of its study will be presented. [Lånad abstract av en liknande rapport]

		\lipsum[1-2]
	\end{abstract}

	\begin{flushleft}
		{\small {\bf Keywords:} keyword A, keyword B,...}
	\end{flushleft}

	% Return to Swedish as language
	\selectlanguage{swedish}

	\tableofcontents

	\clearpage

	\section{Introduktion}
	\subsection{Syfte}
%[Jag tror första 2 styckena av syftet passar mer som abstract fast på engleska] 
%Denna rapport är delad i två delar. Diskussionen av muskotnöt samt dess innehåll, och den möjliga påverkan dess kemiska innehåll har.
%Den psykotropiska berusningen (intoxikationen??) kommer bli uppdelad i två områden i sin presentation. Den första är en beskrivning av plantan; en presentation av metoderna och procedurerna för isolering och identifikation av innehållet från plantans olja, och en noggrann förklaring av de beståndsdelar som mest troligt är involverade i det intoxikativa syndromet (kan man säga så? (intoxicative syndrome)) 

%Utsträckningen av dessa beståndsdelar till dess motsvarande amfetaminer, dess effektivitet i människor, samt dess sannolikhet att vara en godtagbar förklaring av effekterna av muskotnöt, kommer utgöra den senare delen av detta arbete. Det kommer även finnas en del som tar upp muskotnötens historia i början av arbetet, samt en beskrivning samt utsträckning av dess användande i diverse kulturer.

Jag kommer till att presentera en saklig beskrivning av de diverse ämnen som har funnits till att utgöra den flyktiga (och den förmodade aktiva \cite{shulgin1967chemistry}) fraktionen av muskotnöt. Hypotesen lyder så att en eller flera beståndsdelar av muskotnöten är tilldelad rollen för de berusande effekterna av muskotnöt, därför behövs en exakt kemisk definition av muskotnöten. Men före detta, så måste vi definiera just vad som namnet muskotnöt betyder i botaniska termer.










%Syftet för denna rapport är att visa och förklara de psykoaktiva effekterna hos muskotnöt samt dess användningsområden utöver användningen som en krydda som återfinns i många svenska hem.
%Rapporten kommer att göra en uppställning av de eteriska oljorna i muskotnöt för att sedan göra en analys av de olika oljorna för att finna samband till andra kända psykoaktiva substanser.
	\subsection{Bakgrund}
Muskotnöt, kommer ifrån trädet \textit{Myristica fragrans}, som ursprungligen är ifrån de Indonesiska Bandaöarna (även känt som kryddöarna). Legenden har det att när \textit{M. fragrans} blommar så får den överväldigande doften av nötterna fåglarna att falla till marken. \cite{kreig1965green}
Detta kan ha mer att göra med de narkotiska egenskaperna av muskotnöten än dess doft.

Trots att invånarna av Bandaöarna inte använde muskotnötterna som en krydda så finns det underlag på att det använts som en medicin och krydda i Indien och i Mellanöstern så tidigt som 700 f.v.t., \cite{kalbhen1971nutmeg}, dess terapeutiska användningsområden har uppmärksammats av Arabiska läkare sedan 700-talet. \cite{weil1967nutmeg} Muskotnöt anlände inte i Europa förens Medeltiden och det finns motstridande källor på om det var Arabiska handelmän eller återvändande korstågsriddare som tog med sig kryddan till Europa. Muskotnöt var något av en exotisk handelsvara förens 1600-talet då Portugiserna upptäckte Bandaöarna.

Efter denna upptäckt förlorade muskotnöten dess status som exotisk handelsvara. På höjden av dess värde, bars muskotnöt av Européer som ett tecken på rikedom.

\subsubsection{Muskotnöt som krydda}
Den välkända nötens vanligaste användningsområde är som en krydda. Muskotnöten producerar även en krydda känd som muskotblomma, vilket är det brungula mjuka hölje som omger kärnan. Muskotblomma är en vanlig krydda i äldre svenska recept, och fortfarande i många norska. Den är något lenare i smaken är muskotnöt men båda används på liknande sätt i matlagning. Muskotblomma innehåller samma oljor som gör muskotnöten psykoaktiv. \cite{entheogenreview}

De två kryddorna var som populärast på 1700-talet i England, där de användes som krydda till många olika rätter, som
roast mutton, stewed pork, pies, puddings, and cordials. Muskotnöt samt muskotblomma
har även använts för att krydda mång andra rätter, som soups, gravies, milk products, fruit juices,
sweet sauces, gelatins, alcoholic beverages, snack foods, and breakfast cereals; [källa här eller?] \textbf{Fixa detta till svenska}
Trots Muskotnötens tidigare utbredda användning inom matlagningen och dess plats i de flesta kryddskåp så har dess användning förminskats till enstaka kryddning av pajer, kakor samt av äggtoddy. \cite{entheogenreview}


\subsubsection{Muskotnöt som medicin}
Sedan samma tid som muskotnöten blev populär som krydda så har den även
varit använd som med medicin.
Muskotnöten har använts för helande (medicinska?) syften runtom i världen,
sedan den introducerades till Europa och västvärlden så har
dess medicinska användningsområden att användts av Europeiska doktorer.
Trots att muskotnöten användes för väldigt många olika syften inom sjukvården
så finns det ett antal som är mer
värdiga att nämnas på grund av dess utbredda användning.

Nutmeg has been used to treat rheumatism in Indonesia, Malaysia, England, and China. The
essential oil is used externally to treat rheumatic pains, limb pains, general aches, and
inflammation. In England, far into the twentieth century, a nutmeg was simply carried in one's
pocket to ward off the pains of rheumatism (Rudgley 1998).


Nutmeg is probably most widely used to treat stomach complaints. It has been used in South
East Asia, India, the Middle East, and Europe to treat stomach aches and cramps, to aid digestion, and to dispel gas.


Perhaps the most infamous medical use of nutmeg, as mentioned earlier, is as an
abortifacient. It is not clear how far back this use dates, but it was a popular--albeit ineffective-- “remedy” at the end of the nineteenth century and beginning of the twentieth century.

While there doesn't appear to be any traditional use of nutmeg as a mood elevator, several individuals have noted that it does indeed have such properties. The German writer Georg
Meister noted nutmeg's uplifting effects in his 1692 work Der Orientalisch-Indianische Kunst-
und Lust-Gärtner (Oriental-Indian Art and Pleasure Gardener) commenting that "it can greatly
refresh even the ill and cheer them up with fresh spirits" (Rätsch 2005); and the twelfth century
mystic Hildegard von Bingen had this to say:

\begin{displayquote}

When a human being eats nutmeg it opens his heart, and his sense is pure, and it
puts him in a good state of mind. Take nutmeg and (in the same amount) cinnamon
and some cloves and grind them up. And then, from this powder and some water,
make flour--and roll out some little tarts. Eat these often and it will lower the
bitterness of your heart and your mind and open your heart and your numbed
senses. It will make your spirit happy, purify and cleanse your mind, lower all bad
fluids in you, give your blood a good tonic, and make you strong \cite{ratsch2006pagan}

\end{displayquote}

Nutmeg is still used in Arabic and Indian folk medicine today, but its use as an herbal remedy in
Europe is long forgotten. Use as a medicine never seems to have caught on in the United
States, with the exception of its use as an abortifacient in the nineteenth century.



\subsubsection{Muskotnöt som afrodisiakum}

Ett mindre känt användningsområde av Muskotnöten är som ett afrodisiakum, vilket på vardaglig svenska betyder
kärlekselixir.
I Indien så har Muskotnöten använts till curry maträtter men även tuggbuss för dess stämnings höjande effekter \cite{ratsch2005encyclopedia}

Medan användningen av Muskotnöten som ett afrodisiakum i Europa inte verkar vara något välkänt eller utbrett så
finns det enstaka exempel. William Salmon, en 1600-tals Engelsman skrev 1693, i ett själv-exeperiment där
Muskotnötsolja gnuggat på könet producerat sexuell lust \cite{RudgleyR}.
Mest nämnvärt är nog en gammal Tysk folk tradition, där en flicka ska svälja en hel muskotnöt, samla den hela nöten
efter passage, göra den till ett puder och ha i maten av dess älskade. Att göra detta ska enligt traditionen
få mannen i fråga att förälska sig i flickan. \cite{ratsch2005encyclopedia}

%Finns en studie där de använt råttor som test subjekt kan ta med den om jag orkar.


\subsubsection{Muskotnöt som dröm förbättrare}

Det finns inte mycket underlag angående hur muskotnöt samspelar med drömmar. Många experiment har beskrivit effekterna
av muskotnöt till dröm liknande kvalitéer och levande dagdrömmar. \cite{entheogenreview}
(Bilagor här www.erowid.com)

Den mest kompletta rapporten angående muskot nötens effekter på drömmar kommer ifrån Paul Devereaux, han intog
två teskedar av riven muskot och strödde ut den essentiella oljan av muskot på sin kudde samt lakan i en del
av ett själv experiment. Devereaux rapporterade att han blev helt självmedveten under drömmen där han flög igenom
en tunnel i en hög hastighet. Devereaux fann även att sina taktila sinnen var delvis funktionella i drömmen.
Devereaux beskrev hur han ryckte till sig löv från träd han passerade och även kände motkraften från grenarna samt
bladverket som grävde ner sig i hans hand. \cite{RudgleyR}

Devereauxs rapport stärker påståendet att muskot kan ha en effekt på klarheten av drömmar samt dröm återkallning,
likväl, mer definitivt underlag för dessa påståenden saknas.


\subsubsection{Muskotnöt som berusningsmedel}

Historisk sett har Muskotnöt använts i Egypten som ett surrogat för hashish. Det har även används i Indien, antingen
tuggat, eller snusat tillsammans med tobak, eller med tuggtobak, men det finns lite information kring dessa administreringsvägar.

Muskotnöt introducerades först som en krydda i Europa och senare som en medicin. Europeerna fortsatte dock att ignorera denna populära kryddas berusande effekter i flera århundranden.

Det första dokumenterade fallet av muskotnöt som källan till en berusning finner vi år 1576 när en gravid brittisk kvinna drabbades av yrsel efter hon intagit mellan tio och tolv muskotnötter. \cite{stein2001nutmeg}
Skulle det inte varit för ryktet av muskotnötens effektivitet som ett abortmedel så skulle dess
psykoaktiva egenskaper troligtvis förblivit okända för en lång tid.
Enskilda fall av muskotnöts förgiftning var senare publicerade, men muskotnötens berusande egenskaper förblev mestadels obemärkta och outforskade.

I det sena 1800-talet och tidigare 1900-talet så blev muskotnöten återigen populär som ett abortmedel. I och med detta ökade fallen av muskotnöts förgiftning och fler fallstudier rapporterades.
Detta hjälpte till att måla en klar bild av muskotnötens verkan och effekter.
Det är inte säkert hur muskotnöten kom till att bli en rekreations-drog, men det verkar ha sitt
ursprung i det tidiga 1900-talet när dess användning uppstod i USAs fängelsen som ett alternativ
till marijuana och andra otillåtna substanser.
Vissa författare tyder på att muskotnötens användning som narkotika inte uppstod förens efter
andra världskriget. Dock så rapporterar Malcolm X i sin autobiografi att det redan funnits en kultur kring
användandet av muskotnöt vid Charlestown state Prison år 1946, detta tyder på att fångarna redan
varit bekända med muskotnötens narkotiska effekter för en viss tid.
Malcolms X ord löd såhär i sin autobiografi, utgiven 1965:
\begin{displayquote}

I first got high in Charlestown on nutmeg. My cellmate was among at least a
hundred nutmeg men who, for money or cigarettes, bought from kitchen worker
inmates penny matchboxes full of stolen nutmeg. I grabbed a box as though it
were a pound of heavy drugs. Stirred into a glass of cold water, a penny
matchbox full of nutmeg had the kick of three or four reefers \cite{malcolmx}.

\end{displayquote}

Efter utgivandet av Malcolm X sin autobiografi så kom intresset av muskotnöt som ett berusningsmedel återigen i liv och intresset har levt kvar tills nutid.
Användandet av muskotnöt i fängelsen blev så småningom så utbredd att muskotnöt blev helt bortplockat ifrån fängelseköken.


\subsubsection{Effekter av Muskotnöt}

Psykologiska effekter av muskotnöt innefattar, torr mun, illamående, hjärtrusning, rodnad,
domnandet av lemmar, hypotension (lågt blodtryck), eufori, avskildhet, CNS-excitation,
hallucinationer samt andningssvårigheter.
Muskotnöt har ingen märkvärd effekt på storleken av pupiller. \cite{entheogenreview}

Muskotnöt är bäst beskriven som en deliriant. I låga doser delar den karaktären av en kombination av alkohol och marijuana. I högre doser är effekterna mer lika de hos tropana alkaloider, ex. scopolamine och orsakar förvirring, disorientation samt hallucinationer.
Effekterna av muskotnöt kommer och går i vågor. I ena stunden kan det vara en känsla av onykterhet och i andra stunden kan den känslan ha avtagit.
Som effekterna avtar så blir avgränsningen mellan en vanlig och en icke vanlig verklighet tunn och tillåter användaren att övergå mellan de två med någon form av kontroll.

En anledning till att effekterna av muskotnöt bibehåller sin mystik för många är att muskotnötens berusning följer en unik tids-linje. Detta är även möjligtvis anledningen till fientligheten som finns mot muskotnöt som ett berusningmedel. Många antar att muskotnöt kommer producera effekter som utvecklar sig inom en timme vilket de gör med de traditionella psykadeliska drogerna som psilocybin-innehållande svampar eller LSD. Vilket det inte gör, användaren tror att den inte tagit tillräckligt då den inte upplever några önskade effekter och ökar dosen. Detta leder till en oavsiktlig överdosering och en berusning utöver den först önskade. (För att bäst beskriva muskotnötens berusning och för att undvika missöden så har jag delat upp dem i olika steg och summerat effekterna som kan upplevas vid respektive steg.(tror jag hoppar detta, skriver kolla bilaga))

\\

%\paragraph{Inledande berusning (timmar 4-8)}


%\paragraph{Topp berusning (timmar 8-12)}


%\paragraph{Slut av topp (timmar 13-18)}


%\paragraph{Rest berusning (timmar 19-25)}


%\paragraph{Sista steget-baslinje (timmar 26-32)}


\subsubsection{Dosering av Muskotnöt}
Hur potent muskotnöten är kan variera kraftigt mellan en nöt och en annan;
man bör vara medveten om sitt stoff innan man intar en större dos.
Muskotnötter från Östindien sägs vara mer potent än de producerade i Västindien, och färskt
mald muskot är även känt att vara mer potent än för mald muskot.\cite{entheogenreview}
Muskotnöt är inte att utsättas till justeringar av dosering i efterhand pågrund
av dess sena utveckling av effekter, detta gör att en kännedom om potens viktig.
\cite{entheogenreview}

\\

Den följande informationen angående dosering är baserade på en analys av 176 erfarenheter publicerade online på Erowid.com

\paragraph{Tröskel (3-5 gram eller 1-1.5 tsk)}
En tröskel dos av muskotnöt utmärks av en eufori, avslappning, humörs höjning, glädje samt
förstärkning av sinnena.
Vissa personer upplever inga effekter vid denna nivå.

\paragraph{Låg-Måttlig (6-10 gram eller 1.5-3 tsk)}
En låg-måttlig dos av muskotnöt kan producera mer distinkta effekter än en tröskel dos, kan framkalla visuella förvrängningar, CEVs, samt hörselhallucinationer.
Korttidsminnet kan försämras och tal kan bli något sluddrigt under toppen av en
låg-måttlig dos.

\paragraph{Måttlig (11-15 gram eller 1-1.5 msk)}
En måttlig dos av muskotnöt kan orsaka sluddrigt tal, desorientering, och förlust av koordination. Tidigare nämnda effekter ökas och brukaren kan uppleva milda visuella fenomen.

\paragraph{Måttlig-Hög (16-20 gram eller 1.5-2 msk)}
En måttlig-hög dos kan producera ett vaket dröm likt tillstånd. En individ sökte intensiv vård efter ett intag av 15-20 gram av muskotnöt. Individen rapporterade andningssvårigheter, medvetslöshet, inbillningar samt panik.

\paragraph{Hög (20-25 gram eller 2-2.5 msk)}
En hög dos kan fortsatt förstärka inbillningen att användaren befinner sig i en dröm värld. Användaren kan uppleva magont.

\paragraph{Icke rekommenderad (25+ gram eller 2.5+ msk)}
Doser vid denna grad ökar generellt sett inte de psykoaktiva effekterna av muskotnöt, men kommer mer troligt att öka tidsomspannet av intoxikationen (berusningen) och vill med det ta längre att återhämta sig från.
Fysiskt obehag som magont, hjärtklappningar, illamående samt yrsel tenderar att öka.
Kräkningar inträffar sällan.
Användaren kan uppleva andningssvårigheter eller svårheter med urinering. Användaren kan också få vanföreställningar.
Utav 66 individer som rapporterat ett intag av över 25 gram av muskot, 17\% rapporterade en svår upplevelse och 45\% av dom uppsökte akut-sjukvård.
Den genomsnittliga dosen för de som rapporterade negativa effekter var mellan 29 och 30 gram, dock var median dosen endast 25 gram. Genomsnittsdosen för de som uppsökte akut-sjukvård var 47.5 gram, medans median dosen var 52.5 gram. Med den ostadiga potensen hos muskotnöt så kan vissa individer krävt en högre dos för att uppnå måttliga effekter, men man bör vara extremt bekant med potensen hos sitt stoff innan intag av en hög eller icke rekommenderad dos.


%Några korta ord först sedan för mer utförligt i bilaga \#x







%Muskotnöt har en lång historia av mänsklig användning utöver att vara en krydda vilket den tillgängliga litteraturen visar på. Muskotnöt har använts till att påtvinga abort, återinsätta en missad menstruation, självmordsförsök, bota skallighet samt som ett berusningsmedel \cite[31-32]{shulgin1995pihkal} där det sistnämnda är vad denna rapport kommer undersöka i detalj.
%Muskotnöt består av en uppsjö av olika Eteriska oljor vilka denna rapport ska undersöka i ett försök att förklara muskot nötens psyko aktiva effekter in vivo.
%Dessa Eteriska oljor kommer undersökas i denna rapport i ett försök av att förklara muskotnötens verkan på psyket.

	\subsection{Frågeställningar}
		\lipsum[3-4]
	\section{Metod}%

\subsection{Litteraturstudie}

	\subsection{Material}
%		\lipsum[7-8]
	\subsection{Avgränsningar}
Avgränsningar som arbetet påverkas av är att data ej kan tillverkas på plats utan
är lånad från andra institutioner på grund av brist av verktyg samt resurser.
Detta är förväntat från en rapport på denna nivå och bör ej ses som en brist.

\pagebreak

\section{Disposition}

Dispositionen av detta arbete kommer bestå av en analys av muskotnötens flyktiga oljor
vars verkan kommer tas upp i resultatet.

Det måste tas upp här hur det finns två klassiska sätt att extrahera de potentiellt intressanta beståndsdelarna av den hela nöten.

\textbf{Fig. 1} \cite{shulgin1967chemistry} visar en ungefärlig distribution som kan förväntas vid användningen av dessa metoder. Processen av extraktion med ett organiskt lösningmedel ger ungefär en tredjedel av den ursprungliga vikten. Denna fraktion är känd som de icke flyktiga oljorna, även känt som smöret av muskotnöt eller ''Oleum Myristicae expressum''. Denna fraktion är väsentligen fri från flyktiga oljor, alltså de förmodade aktiva och består till störst del av triglycerider. Myristinsyra är den huvudsakliga beståndsdelen här, dock finner vi både oljesyra samt linol syra. Den lilla icke fett resten består av icke-sapofinerbara (unsaponifiable) beståndsdelar, mestadels syresatta polyterpener och phytosteroler. \cite{shulgin1967chemistry}

Resultatet från den krossade nöten till ångdestillation avlägsnar från 10 till 15\% av vikten, känd som den flyktiga olje fraktionen.

Överlappningen som visas med den uttryckta fraktionen beror på faktumet att vissa av de flyktiga beståndsdelarna är avlägsnade i lösningsmedelsextraktionen och hålls hårt av de närvarande fasta beståndsdelarna.
Denna flyktiga fraktion består till mestadels av terpener vilket utgör ungefär 80\% av dess totala vikt.

Resten är den aromatiska fraktionen, bestående av etrar och fenol kroppar.\cite{shulgin1967chemistry}


\includegraphics[scale=0.103]{Figure1}


Det som blir kvar efter dessa processer och extraktioner av de flyktiga oljorna utgör till ett ungefär 50\% av den ursprungliga vikten av muskotnöten. Den är förmodligen en cellulosa liknande massa, och den förblir komplett outforskad i form av någon kemisk analys.

Det måste bli fastställt här hur i väntan på kommande diskussion angående farmakologin av muskotnöt att ingen definitiv utvärdering av dessa fraktioner (fetterna och massan) har blivit gjorda. Det är dock, generellt accepterat att det är den flyktiga oljans fraktion som man måste vända sig till för de aktiva beståndsdelarna av muskotnöten, och det är denna ''Olja av Muskotnöt'' som är erkänd av den Amerikanska Farmakopén som en medicin.

Denna flyktiga olja innefattar mellan en åttondel och en tolvtedel av den hela nöten och är saken för denna studie.

\hbox{\hspace{-0.1cm}\includegraphics[scale=0.75]{GCgraph}}

\begin{minipage}[t]{2cm}
\flushleft
\textsc{\textbf{Fig. 2}}
\end{minipage}
\hfill
\begin{minipage}[t]{2cm}
\flushright
\textsc{\textbf{Fig. 3}}
\end{minipage}


\begin{table}[!htbp]
\small
\centering
\caption{GC analys av muskotnötens flyktiga olja vid 90 bar - $23\,^{\circ}\mathrm{C}$}
\begin{tabular*}{\textwidth}{l @{\extracolsep{\fill}} @{}cll@{}}
\toprule
Topp nummer & \ Retentions tid (min) & Yta (\%) & Identifikation                                      \\ \midrule
1           & 2.880                & 11.0520   & $\alpha$-pinene                                      \\
3           & 3.898                & 10.1872   & $\beta$-pinene                                       \\
4           & 4.175                & 36.6348   & sabinene                                            \\
6           & 4.763                & 3.1476    & myrcene                                             \\
8           & 5.470                & 3.7710    & limonene                                            \\
9           & 5.655                & 2.2727    & $\beta$-phelandrene                                  \\
10          & 6.545                & 1.8826    & $\gamma$-terpinene                                   \\
11          & 7.097                & 1.6559    & cymene                                              \\
14          & 11.287               & 1.8101    & \textbf{cis*}   \\
17          & 12.658               & 1.7945    & \textbf{trans*} \\
21          & 13.455               & 3.6766    & terpinen-4-ol                                       \\
30          & 16.907               & 1.0075    & safrol                                              \\
36          & 20.907               & 3.2933    & elemicin                                            \\
37          & 21.393               & 6.9835    & myristicin                                          \\ \bottomrule
\end{tabular*}
\note{\textbf{cis*} = 1-methyl-4(1-methylethyl)2-cyclohexen-1-ol (cis)
\\
\textbf{trans*} = 1-methyl-4(1-methylethyl)2-cyclohexen-1-ol (trans)}

\end{table}

En bestämmande analys av denna flyktiga olja har lånats av \cite{spricigo1999extraction}.
Till denna analys användes hela muskotnötter som levererats av Bretzke Alimentos \textit{(Jaraguá do Sul, SC, Brazil)}. De maldes med en kaffekvarn, efter det utfördes
en extraktion av de flyktiga oljorna. Extraktionen utfördes med en ångdestillation enligt metoden beskriven av ``The American Spice Trade Association'' för bestämmandet av flyktiga oljor
\cite[citerad av Ferreira]{spricigo1999extraction}.
\textbf{Fig. 2} visar en GC analys av den hela flyktiga olje fraktionen erhållen
från proceduren. Tabell 1 visar en identifikation av de större topparna.
Från detta resultat kan det visas hur huvudämnena av den flyktiga oljan är $\alpha$-pinene, $\beta$-pinene, sabinene samt myristicin, den karaktärsgivande aromatiska beståndsdelen av muskotnöt.

\textbf{Fig. 3} utgår, det är en identifikation av de fett syrorna som återfinns i den icke flyktiga olje fraktionen av muskotnöt, tillhörande tabell ingår ej i denna rapport men kan finnas i \cite[s.258]{spricigo1999extraction}

Den mindre gruppen som vi ser i \textbf{Fig. 1}, den aromatiska
eter fraktionen är den mest intressanta av de olika vilket kommer visas senare
och är mer trolig att vara involverad i de psykoaktiva effekterna av
muskotnöt.
De tre huvudkomponenterna av den aromatiska fraktionen är myristicin, elemicin
och safrol vilka står för nästan 9/10 av fraktionen.
I tidigare studier av muskotnöt så har myristicin alltid erkäns som en
huvudbeståndsdel och har därefter trots vara ansvarig för de berusande effekterna.

Uppgiften att tillgodose kemiska strukturer till de olika beståndsdelarna är
en enkel uppgift i jämförelse med uppgiften att tillse ansvar för de
beståndsdelar som bidrar till de berusande och psykotropiska egenskaperna hos muskotnöten.
Kärnan i sig är den enda delen av nöten med ryktet av biologisk aktivitet.
Vidare kan man påstå att den psykoaktiva beståndsdelen eller beståndsdelarna (compound, compounds)
troligtvis återfinns hos den flyktiga olje fraktionen av kärnan, vilket kan styrkas
med att djurstudier (toxiologi studier) som visat att den innehåller samma
verkan som den hela nöten.
Mänskliga experiment med riven muskotnöt tömd på sina flyktiga oljor har
misslyckats med att visa någon form av psykofarmakologiskt svar. (låter konstigt ska ändra) \cite{truitt}


Med en lyckad tilldelning av kemiska strukturer till de olika beståndsdelarna
som kan ansvariga för effekerna, så måste man undersöka hur var och en, eller
mer troligt tillsammans, kan uppnå rollen som en rimlig förklaring till
effekterna vi kan få av den hela nöten.
Vi har två typer av ämnen att betrakta, terpenerna och de aromatiska etrarna.
Det är frestande att direkt utesluta terpenerna fastän de utgör en överväldigande
del av den flyktiga oljan. Terpenernas kolväten är generellt kända som att vara
biologiskt effektiva endast som retmedel.
Terpentin har en sammansättning lik denna terpen fraktion; det har använts
i många huskurer men har visserligen inget rykte som ett berusningsmedel.
Det kan å andra sidan ha någon funktion vid upptagning av olika aromatiska etrar i
magsäcken.


Den aromatiska fraktionen är den som verkar vara den mest troliga källan till de
berusande effekterna av muskotnöt. \textbf{Tabell 2} visar strukturerna av varje
förening återfunnen i den aromatiska fraktionen. Även visat är mängden i milligram
som skulle återfinnas hos 20g av en hel muskotnöt, det antaget vara dosen som krävs
för berusande effekter. (skriv om sista meningen)

\pagebreak%		\lipsum[13-14]

\centered{\caption{\textbf{Tabell 2}}

\hbox{\hspace{-1.83cm}\includegraphics[scale=0.12]{Tabell2}
}

Som fastställdt innan så upptar safrol, myristicin, och elemicin 84\% av den
aromatiska fraktionen, och är därefter de huvudsakliga ämnena som vi kommer att överväga.
Dock måste man ha i åtanke om möjligheten att en av de mindre beståndsdelarna
kan ha en ovanligt hög potens och bidra till effekterna.

Av de huvudsakliga beståndsdelarna så är myristicin överlägset den mest förekommande,
av denna anledning så har tester utförts specifikt för psykotropisk effekt av
Truitt, \textit{et al} \cite{truitt}. Doser på 400mg myristicin, nästan det dubbla som förväntas
hos 20 g. av typisk muskotnöt utdelades till frivilliga människor (vad heter detta, svenska) och de observerade symptomen var åtminstonde tydande av psykotropiska effekter hos 6 av 10
försöks personer.

Safrol, även det en beståndsdel av andra naturliga oljor och kryddor, mest känd från oljan från Sassafras trädet vilket safrol utgör 80\% av. Både oljan samt det utvunna Sassafrasteet har använts i en bred utsträckning, måttligt som ett smakämne, och i högre mängder som intern medikament; men varken har något rykte för någon psykotropisk aktivitet vilket muskotnöten har.

Elemicin, är ovanligt att finna bland smaksättande oljor och kryddor, ändå så
återfinns det i en märkvärdig mängd i muskotnöt. Vidare, vilket kommer tas upp i diskussionen så varierar mängden av elemicin kraftigt och beror på ursprunget av nöten. Elemicin återfinns även bland ett antal obskyra eteriska oljor, inga med någon känd farmakologisk användning. Elemicin är vidare inte separerbar från myristicin vid fraktionell destillation. Myristicinet som använts i all tidigare farmakologi (inklusive de mänskliga experimenten ovan) har ursprungat från destillation av muskotnöts olja och trotts att vara bestående av endast myristicin. Därav så kan vi inte veta om det innehöll elemicin.
Skillnaderna mellan innehållet av elemicin från olika ursprung kan möjligtvis redogöra för den hög diskrepans som återfunnits hos de rapporterade effekterna hos muskotnöt, vilket i sin tur antyder att elemicin kan visserligen vara en aktiv beståndsdel.

Av den aromatiska fraktionen, i mindre beståndsdelar finner vi endast, eugenol och isoeugenol som tidigare använts antingen som smakämnen eller medicin.
De utgör ungefär 80\% av nejlikolja exempelvis, men sökning bland litteraturen angående sådana naturliga produkters rykte som berusning eller missbruksmedel har visat sig vara meningslöst.

Därav finns det flera möjligheter genom vilka de aromatiska beståndsdelarna kan vara involverade som en källa till effekterna.
\begin{enumerate}
	\setlength\itemsep{0em}
\item En av de beståndsdelar som är närvarande i endast väldigt små mängder har en ovanligt hög potens.
\item Elemicin kan vara en större källa av aktivitet än tidigare trott, eller
\item En kombination av två eller fler av aromaterna närvarande är involverad.
\end{enumerate}





%

%dessa resultat verkar visa en stor diskrepans mellan muskot från olika geografiska källor, gör en komparativ analys, citera shulgin.


%Nu ska jag börja gå in på vad den flyktiga oljan består av och vissa likheter mellan myristicin samt tma osv, norsk källa här.

%Ska infoga figur 2 vilket ska vara en ms/gs graf och förklara den.

%Jag ska komma fram till att muskokot = myristicin/elemicin innan jag börjar med resultatet.

%Vad muskotnöt består av här, grafer osv1

	\subsection{Resultat}

	De tre ämnen som närvarar i störst mängd, myristicin, elemicin och safrol kan vara en tillräcklig förklaring till muskotnötens berusande effekter.

	Det är värt att påpeka att muskotnöt är den enda växtkällan inom vilken dessa
	tre ämnen förekommer tillsammans i någon avsevärd mängd, och vilket som kommer visas senare bidrar med något annorlunda delar av den totala effekten.

	Ringsubstitutionen för dessa ämnen från den aromatiska fraktionen är märkvärdiga då ett flertal av dem, speciellt myristicin, elemecin och safrol, vilka är identiska till ring strukturerna av ämnen med en känd och etablerad psykoaktiv effekt. Den allyl sido kedjan kan enkelt ändras med hjälp av kemisk modifikation, vilket är visat i Fig. 4, detta skulle kunna omvandla de naturligt förekommande ämnena till de med känd psykoaktiv effekt.

	Det har föreslagits (8 shulgin), att \textbf{the in vivo addition
	of ammonia to the olefinic (alken) site in either the allyl or the propenyl isomer would yield amphetamines directly}.
	Men att tala of amfetaminer som en kemisk klassifikation vore inte helt korrekt, men vi använder ordet för att hänvisa till diverse
	\textbf{methoxylated phenyl-isopropylamines} (phenetylaminer).
	``RO'' gruppen i figuren visar på närvarandet av någon form av ester på ringen,
	och inkluderar därefter alla de aromatiska estrarna från oljan av muskotnöt men även
	från många andra naturliga oljor.
	Den troliga mekanismen för en sådan \texit{in vivo} modifikation har beskrivits,
	och är trolig till den grad att var och en av reaktionerna utförts \textit{in vitro}.

	Stöd för påståendet om att en modifikation som denna skulle kunna ske in vivo har erhållits
	av Barfknecht (9 shulgin), som funnit bevis på produktionen av amfetaminer i råttor
	efter att ha matat dem med allylbenzene. Detta motsvarar tillskottet av
	ammoniak i Fig. 4 utan ``RO'' ester gruppen.




	\hbox{\hspace{0cm}\includegraphics[scale=0.1125]{Figure4}
	}
	\begin{center}\textbf{Fig. 4.}\end{center}


	En studie av de möjliga \textit{in vivo} amineringarna av dessa ring substituerade
	ämnen har utförts, de amfetaminer som skulle vara resultaten av en sådan
	aminering har blivit fastställda för de respektive ämnen som återfunnits
	i den aromatiska fraktionen \textbf{har införts i en tabell, skapa denna tabell}

	Den bas som korresponderar till safrol, är 3,4-metylendioxiamfetamin, MDA.
	Denna bas var först beskriven farmakologiskt av Gordon Alles (11, shulgin) som
	rapporterade visuella effekter vid en dos på 120mg. Efterföljande experiment
	(12, shulgin) på ett mer omfattande antal försökpersoner visade milda, om några
	visuella effekter, men snarare en emotionell verkan, vilket har visat sig
	vara av ett betydande värde för psykoterapi.

	Den bas som skulle vara resultaten av en aminering av myristicin, är
	3-methoxy-4,5-methylenedioxyamphetamine, MMDA.

	Den bas som skulle resultera från en aminering av elemicin, är \\
	3,4,5-trimethoxyamphetamine, TMA. Detta har varit ett känt psykoaktivt medel
	för en viss tid (14 shulgin, pihkal).
	Den har på olika sätt blivit beskriven som kapabel till hallucinogena effekter
	och lett till en synes ovänlig respons. (pihkal)

	


	\subsection{Diskussion}





\pagebreak
%	\section{Bilagor}
%\subsection{Bilaga 1}
%\input{bilagor/bilaga1.tex}
%\subsection{Bilaga 2}
%\input{bilagor/bilaga2.tex}
%\subsection{Bilaga 3}
%\input{bilagor/bilaga3.tex}
%	\pagebreak


\printbibliography



\end{document}
